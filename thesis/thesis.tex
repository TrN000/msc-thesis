\documentclass{amsbook}
\usepackage[backend=biber,style=authoryear]{biblatex}
\usepackage[most]{tcolorbox}

\newtcolorbox{redbox}{%
  colframe=red,
  colback=white,
  boxrule=0.8pt,
  fonttitle=\color{black}\bfseries,
  title="unfinished"
}


% Define theorem environments
\newtheorem{theorem}{Theorem}[section]   % numbered within sections
\newtheorem{lemma}[theorem]{Lemma}       % shares counter with theorem
\newtheorem{proposition}[theorem]{Proposition}
\newtheorem{corollary}[theorem]{Corollary}

% Non-italic environments (e.g. Definition, Example)
\theoremstyle{definition}
\newtheorem{definition}[theorem]{Definition}
\newtheorem{example}[theorem]{Example}

% Remark-style (upright text, no bold title)
\theoremstyle{remark}
\newtheorem*{remark}{Remark}   % the * makes it unnumbered

\newcommand{\Z}{\mathopen{}\ensuremath{\mathbb{Z}}\mathclose{}}
\newcommand{\C}{\mathopen{}\ensuremath{\mathbb{C}}\mathclose{}}
\newcommand{\R}{\mathopen{}\ensuremath{\mathbb{R}}\mathclose{}}

\addbibresource{./references.bib}

\title{Proof Assistants and Proof Formalization}
\author{Nicolas Trutmann}
\date{January 2026}

\begin{document}

\maketitle

\tableofcontents

\section{Introduction}\label{sec:introduction} % (fold)

\begin{redbox}
  what is a proof assistant, why do proofs "hold" in a proof assistant, are there
  options, which option have we chosen and why, what are we doing with it,
  (eventually) what have we gotten out of this endeavor

  \cite{avigad_foundations_2021}
\end{redbox}

% section Introduction (end)

\section{Sylvester's Theorem}\label{sec:sylvester_s_theorem} % (fold)

Cite the theorem in the form that we prove, big picture overview of the proof,
side by side comparison of "semantic" proof vs "formal" proof (vs formal
proof?)

\begin{redbox}
  \subsection{Prelude}\label{sec:prelude} % (fold)
  definitions and such

  bilinear form, notation
  % subsection Prelude (end)
\end{redbox}


\begin{redbox}
  The original theorem is over 170 years old but remains reasonably legible to
  modern readers. Perhaps less familiar to the modern reader is the
  presentation of the theorem, where the statement is entirely contained in the
  title of the paper, and the body of the paper is devoted entirely to the
  proof of the statement.

  The title reads:
  \begin{quotation}
    A DEMONSTRATION OF THE THEOREM THAT EVERY HOMOGENEOUS QUADRATIC POLYNOMIAL
    IS REDUCIBLE BY REAL ORTHOGONAL SUBSTITUTIONS TO THE FORM OF A SUM  OF
    POSITIVE AND NEGATIVE SQUARES. [sic.]
  \end{quotation}

  % TODO: actually read the proof and see if it holds and is equivalent to the modern statement?


  We'll be following the more modern approach of Lang, whose theorem reads like this:

\end{redbox}

\begin{theorem}[Sylvester]
  Let $E$ be a real vector space with a nondegenerate bilinear form $g$.
  There exists an integer $r\in \Z$, $r \geq 0$, if $\{v_1,\dots,v_n\}$ is an
  orthogonal basis of $E$ then for $r$ of them we have $v_i^2 > 0$ and for
  $n-r$ of them $v_i^2 < 0$.
\end{theorem}

\begin{proof}

  Suppose $v_1,\dots,v_n$ and $w_1,\dots,w_n$ were two orthogonal bases.

  Let $a_i = v_i^2$ and $b_i = w_i^2$,\\
  of which $a_1,\dots,a_r > 0$, and $a_{r+1},\dots,a_n < 0$ \\
  and $b_1,\dots,b_r > 0$, bnd $b_{r+1},\dots,b_n < 0$ for some integers $r,s$.

  It suffices to show that $r=s$.

  To that end we'll show that the set $\{v_1,\dots,v_r, w_{s+1},\dots, w_n\}$
  is linearly independent. Because then we get that $r + (n-s) \leq n$, and
  therefore $r \leq s$ and by symmetry, $r=s$.

  Suppose that

  \[
    (x_1 v_1 + \dots + x_r v_r) + (y_{s+1} w_{s+1} + \dots + y_n w_n) = 0
  \]

  Then

  \[
    x_1 v_1 + \dots + x_r v_r =  - y_{s+1} w_{s+1} - \dots - y_n w_n
  \]

  squaring both sides yields

  \[
    x_1^2 a_1 + \dots + x_r^2 a_r =  y_{s+1}^2 b_{s+1} + \dots + y_n^2 b_n
  \].

  The left hand side is $\geq 0$ and the right hand side is $\leq 0$, and
  therefore 0. It follows that all coefficients are 0, which shows that they
  are linearly independent.

\end{proof}


\begin{redbox}
  The original reference is \cite{Sylvester_1852}.

  It is presented in \cite[Thm.~4.1]{Lang_2002} in a modern way.

  Also \cite[Theorem 10.43]{Norman_1986}, which is what's cited on wikipedia.
\end{redbox}



% section Sylvester's Theorem (end)

\printbibliography


\end{document}
